%This is demonstration document
\documentclass{examcard} 
\usepackage{blindtext} 
\usepackage{parskip} 
\usepackage{verbatim}
\usepackage{amsmath}
\usepackage{graphicx}
\usepackage{float}


\begin{document}

\begin{center}
\section*{Short manual for examcard.cls with examples}
\end{center}


In order to generate a card use the following environment:
\begin{verbatim}
\begin{card}
...
\end{card}
\end{verbatim}

\begin{tabular}{ll}
\begin{minipage}{3.5in}
It looks like this:
\end{minipage}&
	
\begin{card}
\blindtext
\end{card}
\end{tabular}\\

If you wish to set your own size for the cards, use the following commands:
\begin{verbatim}
\setcardwidth{width}
\setcardheight{height}
\end{verbatim}
These commands will override the standard values, which are 70mm (height) and 90mm (width).\\

Question names can be generated with the commands:
\begin{verbatim}
\questname[<class>]{<question>} %with number
\questnamenonum[<class>]{<question>} %without number
\end{verbatim}

Examples:

\begin{tabular}{ll}
\begin{minipage}{3.5in}
\begin{verbatim}
\begin{card}
\questname[Physics]{Maxwell's Equations}
\blindtext
\end{card}
\end{verbatim}
\end{minipage}&
\begin{card}
\questname[Physics]{Maxwell's Equations}
\blindtext
\end{card}
\end{tabular}

\begin{tabular}{ll}
\begin{minipage}{3.5in}
\begin{verbatim}
\begin{card}
\questname{Maxwell's Equations}
\blindtext
\end{card}
\end{verbatim}
\end{minipage}&

\begin{card}
\questname{Maxwell's Equations}
\blindtext
\end{card}
\end{tabular}

\begin{tabular}{ll}
\begin{minipage}{3.5in}
\begin{verbatim}
\begin{card}
\questnamenonum[Physics]{Maxwell's Equations}
\blindtext
\end{card}
\end{verbatim}
\end{minipage}&

\begin{card}
\questnamenonum[Physics]{Maxwell's Equations}
\blindtext
\end{card}
\end{tabular}

\begin{tabular}{ll}

\begin{minipage}{3.5in}
\begin{verbatim}
\begin{card}
\questnamenonum{Maxwell's Equations}
\blindtext
\end{card}
\end{verbatim}
\end{minipage}&

\begin{card}
\questnamenonum{Maxwell's Equations}
\blindtext
\end{card}
\end{tabular}

\begin{tabular}{ll}
\begin{minipage}{3.5in}
Mathematics example:
\end{minipage}&
\begin{card}
\questname[Physics]{Maxwell's Equations}
\vspace{1cm}
$$\mathrm{div}\mathbf{E}=4\pi \rho$$
$$\mathrm{div}\mathbf{H}=0$$
$$\mathrm{rot}\mathbf{E}=-\frac{1}{c}\frac{\partial\mathbf{H}}{\partial t}$$
$$\mathrm{rot}\mathbf{H}=\frac{4\pi}{c}\mathbf{j}+\frac{1}{c}\frac{\partial\mathbf{E}}{\partial t}$$

\end{card}
\end{tabular}\\

\begin{tabular}{ll}
\begin{minipage}{3.5in}
You may also include graphics:
\end{minipage}&
\begin{card}
\questname[Graphics]{Circle}

\begin{figure}[H]
\vspace{1cm}
\centering
\includegraphics[scale=0.15]{IMG1}
\end{figure}

\end{card}
\end{tabular}\\

\begin{tabular}{ll}
\begin{minipage}{3.5in}
And tables:
\end{minipage}&
\begin{card}
\questname[Table]{Numbers}
\vspace{1cm}
\centering
{\begin{tabular}{|l|l|l|}
\hline
 1 & 2 & 3\\\hline
 4 & 5 & 6\\\hline
 7 & 8 & 9\\\hline
\end{tabular}}

\end{card}
\end{tabular}\\

Array of cards:\\
\begin{card}
\questname[Array]{Lorem}
\blindtext
\end{card}
\begin{card}
\questname[Array]{Ipsum}
\blindtext
\end{card}\\
\begin{card}
\questname[Array]{Dolor}
\blindtext
\end{card}
\begin{card}
\questname[Array]{Sit Amet}
\blindtext
\end{card}\\

Make sure there is a double backslash \verb=\\= after each row, or some of the cards may be out of the list, like here:\\

\begin{card}
\questname[Array]{Lorem}
\blindtext
\end{card}
\begin{card}
\questname[Array]{Ipsum}
\blindtext
\end{card}
\begin{card}
\questname[Array]{Dolor}
\blindtext
\end{card}\\

Each time \verb=\questname[class]{<question>}= command is executed, it sends data to the list, which can be displayed with the following command:
\begin{verbatim} 
\listgen
\end{verbatim}
Generated list of questions (formed out of the questions used in this manual):
\listgen
Note that the numbering is the same as in cards. If there is no number in the card, so will be in the list.

\end{document}
